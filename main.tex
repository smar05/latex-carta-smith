
\documentclass{book}

\title{Acoples y otros demonios}

\usepackage[utf8]{inputenc}
\usepackage{subfig}         % Multiples imagenes
\usepackage{vmargin}		% Paquete para margenes y formato de página
\setpapersize{A4}
\setmargins	{2.5 cm}     	% margen izquierdo
{0.5 cm}                 	% margen superior
{16.5cm}               		% anchura del texto
{24cm}             		    % altura del texto
{20pt}                		% altura de los encabezados
{1.2cm}               		% espacio entre el texto y los encabezados
{0pt}                		% altura del pie de página
{2cm}                 		% espacio entre el texto y el pie de página

% Preambulo
\usepackage{tikz, tkz-euclide} % hacer gráficos
\usepackage[american, european]{circuitikz} % hacer circuitos
\usepackage{siunitx} % Unidades SI \SI{}{} and \si{} commands for typesetting SI units

\usepackage[spanish, es-tabla, es-noshorthands]{babel}
\usepackage{amsmath}
\usepackage{multirow}
\usepackage{multicol}
\usepackage{tikz, tkz-euclide}
\usetikzlibrary{shapes.geometric, shapes.symbols, arrows, shadows}
\usepackage{graphicx}
\usepackage{float}
\usepackage{booktabs}
\usepackage{circuitikz}
\usepackage{adjustbox}
\usepackage[many]{tcolorbox}
    \tcbuselibrary{theorems}

\usepackage{hyperref}
\usepackage[acronym]{glossaries}
\makeglossaries
%\include{glosario}
\include{acronimos}
\usepackage{biblatex}
\addbibresource{ref.bib}
%\tikzset{>=latex}

\usepackage{pgfplots}
\usetikzlibrary{pgfplots.smithchart}
\usepgfplotslibrary{colorbrewer}
\pgfplotsset{cycle list/Set1-4}
\usepgfplotslibrary{polar}
\pgfplotsset{compat=newest}
\usepgfplotslibrary{smithchart}
\usepackage{siunitx} 
\usepackage{tikz, tkz-base, tkz-fct, pgfplots}
\usepackage{colortbl}
\usepackage{steinmetz}
\usetikzlibrary{bending}
\pgfplotsset{compat=1.13}
\definecolor{micolor}{rgb}{0.9,0.5,0.9}
\ctikzset{current  arrow color/.initial=red}

\usepackage{xparse}
\DeclareDocumentCommand{\newdualentry}{ O{} O{} m m m m } {
  \newglossaryentry{gls-#3}{name={#5},text={#5\glsadd{#3}},
    description={#6},#1
  }
  \makeglossaries
  \newacronym[see={[Glossary:]{gls-#3}},#2]{#3}{#4}{#5\glsadd{gls-#3}}
}


\usepackage{amsmath, amsthm, amsfonts}
\def\RR{\mathbb{R}}
\def\ZZ{\mathbb{Z}}

% De la misma forma se pueden definir comandos con argumentos. Por
% ejemplo, aquí definimos un comando para escribir el valor absoluto
% de algo más fácilmente.
%--------------------------------------------------------------------------

% declaracion de unidades no SI
\DeclareSIUnit{\pf}{pF}
\DeclareSIUnit{\pH}{pH}
\DeclareSIUnit{\nh}{nH}
\DeclareSIUnit{\nf}{nF}
\DeclareSIUnit{\uf}{{\micro}F} 
\DeclareSIUnit{\uh}{{\micro}H} 
\DeclareSIUnit{\ms}{mS}
\DeclareSIUnit{\second}{s}
\DeclareSIUnit{\MHz}{MHz}
\DeclareSIUnit{\dB}{dB}
\graphicspath{ {imagenes/} }

%-----------------------------------------------------------------------------
%               Entorno Para ejemplos 
%-----------------------------------------------------------------------------

 
 %                  Entorno para ecuaciones de ejemplos
 %--------------------------------------------------------------------------
\usepackage{environ}               % Define entornos (ecuaciones de ejemplos)
\NewEnviron{meq}{%
    \begin{equation}
    \scalebox{1}{$\BODY$}           % Se utiliza el paquete "graphix"
    \end{equation}
    }
%------------------------------------------------------------------------------
%                   Entorno de para los diagramas de flujos, como  "Mason"   
%------------------------------------------------------------------------------
\usetikzlibrary{decorations.markings,arrows.meta}
\tikzset
  {midarrow/.style={decoration={markings,mark=at position 0.5 with
     {\arrow[xshift=2pt, purple]{Latex[length=6pt,#1]}}},postaction={decorate}}
  }
  
  \newcommand{\txb}[1]{\small\sffamily #1}
\def\RR{\mathbb{R}}
\def\ZZ{\mathbb{Z}}

%------------------------------------------------------------------------
%           Macros para las redes red1 para grandes (pocos componentes 0-12)
%           pequeños, red2 medianas (varios componentes 0-16)
%           y red3 grandes pequeñas (muchos componentes 0-20)
%-------------------------------------------------------------------------
\newenvironment{red0}[1][0.8]
    {
   \begin{adjustbox}{width=0.4\textwidth}
    \begin{circuitikz}[scale=#1]
    }
    { 
    \end{circuitikz}
   \end{adjustbox} 
    }

\newenvironment{red1}[1][0.7]
    {  \begin{subfigure}[b]{ 0.45\textwidth}
   \begin{adjustbox}{width=0.6\textwidth}
    \begin{circuitikz}[scale=#1]
    }
    { 
    \end{circuitikz}
   \end{adjustbox} 
     \caption{}
    \end{subfigure} 
    }
    
\newenvironment{red2}[1][0.7]{
        \centering
    \begin{subfigure}[b]{0.45\textwidth}
    \begin{adjustbox}{width=0.8\textwidth}
    \begin{circuitikz}[scale=#1]
    }
    { 
    \end{circuitikz}
    \end{adjustbox} 
    \caption{}
    \end{subfigure} 
    }
\newenvironment{red3}[1][0.7]
    {\begin{subfigure}[b]{0.45\textwidth}
    \begin{adjustbox}{width=1\textwidth}
    \begin{circuitikz}[scale=#1]
    }
    { 
    \end{circuitikz}
    \end{adjustbox} 
     \caption{}
    \end{subfigure} 
    }

%------------------------------------------------------------------------
%           Macros para las redes re1 pequeños, rd2 medianos y red3 grandres
%-------------------------------------------------------------------------
\newenvironment{red1a}[1][0.7]{
        
        \begin{adjustbox}{scale=0.8}
        \begin{circuitikz}[scale=#1]
    }
    { 
        \end{circuitikz}
        \end{adjustbox} 
    %\end{center}
    }
    
    \newenvironment{reda}[1][1]
    {\begin{subfigure}[b]{0.4\textwidth}
   \begin{adjustbox}{width=1\textwidth}
    \begin{circuitikz}[x=#1cm,y=#1cm]
    \tikzset{font={\fontsize{15pt}{12pt}\selectfont}}
    %\ctikzset{bipoles/length=25mm}
    \centering
    }
    { 
    \end{circuitikz}
    \end{adjustbox}
    \end{subfigure} 
    }
    
%\ctikzset{bipoles/ammeter/text rotate/.initial=0, rotation/.style={bipoles/ammeter/text rotate=#1},
%}
\newenvironment{redb}[1][0.8]
    {\begin{subfigure}[b]{0.45\textwidth}
   \begin{adjustbox}{width=1\textwidth}
    \begin{circuitikz}[x=0.8cm,y=#1cm]
    \tikzset{font={\fontsize{15pt}{15pt}\selectfont}}
    \ctikzset{bipoles/length=20mm}
    }
    { 
    \end{circuitikz}
    \end{adjustbox} 
     \caption{}
    \end{subfigure} 
    }
%------------------------------------------------------------------------
%          Entorno para los ejemplos
%-------------------------------------------------------------------------
%\usepackage{draftwatermark}
%\SetWatermarkText{\textsc{jpoveda@}} %
% \newtcbtheorem[auto counter, number within = section]{ejemplo}{Ejemplo}%
% 	{   enhanced,
% 	arc=2mm,
% 	interior style={},
% 	attach boxed title to top center= {yshift=-\tcboxedtitleheight/4},
% 	borderline={0.5mm}{0mm}{gray!20!white,dashed},
% 	fonttitle=\bfseries,
% 	fontupper=\itshape,
% 	colbacktitle=white,coltitle=black,
% 	boxed title style={size=normal,colframe=purple!50,boxrule=1pt}
% 	}
% {th}

\newtcbtheorem[auto counter, number within = section]{ejemplo}{Ejemplo CA}%
	{   enhanced,
	arc=2mm,
	interior style={},
	attach boxed title to top center= {yshift=-\tcboxedtitleheight/4},
	borderline={0.5mm}{0mm}{gray!20!white,dashed},
	fonttitle=\bfseries,
	fontupper=\itshape,
	colbacktitle=white,coltitle=black,
	boxed title style={size=normal,colframe=purple!50,boxrule=1pt}
	}
{th}














\begin{document}

\begin{figure}[ht!]
\centering
\begin{tikzpicture}
    \begin{polaraxis}[
                      width=15cm,
                      xshift=-0.5cm, 
                      yshift=-0.5cm,
                      xticklabel style={
                          sloped like x axis={
                              execute for upside down={\tikzset{anchor=north}},
                              reset nontranslations=false,
                          },
                          anchor=south,
                      },
                      xticklabel={\small\pgfmathprintnumber{\tick}\si{\degree}},
                      xtick align=center,
                      grid=none,
                      axis y line = none,]    
   \end{polaraxis}
   
   	\begin{smithchart}
   	[ticklabel style = {font=\tiny},
    width = 14cm,
   % show origin,
    smithchart mirrored,
    grid style={blue!60},
    ticklabel style={blue},  
     xticklabel shift=-10pt,
     ]
    %Curva Admitancia
    \addplot[mark=none,line width=1,,color=orange]
       coordinates{
(	100	,	500	)
(	10	,	50	)
(	8	,	40	)
(	6	,	30	)
(	4	,	20	)
(	2	,	10	)
(	1.8	,	9	)
(	1.6	,	8	)
(	1.4	,	7	)
(	1.2	,	6	)
(	1	,	5	)
(	0.8	,	4	)
(	0.6	,	3	)
(	0.4	,	2	)
(	0.38	,	1.9	)
(	0.36	,	1.8	)
(	0.34	,	1.7	)
(	0.32	,	1.6	)
(	0.3	,	1.5	)
(	0.28	,	1.4	)
(	0.26	,	1.3	)
(	0.24	,	1.2	)
(	0.22	,	1.1	)
(	0.2	,	1	)
(	0.18	,	0.9	)
(	0.16	,	0.8	)
(	0.14	,	0.7	)
(	0.12	,	0.6	)
(	0.1	,	0.5	)
(	0.08	,	0.4	)
(	0.06	,	0.3	)
(	0.052	,	0.26	)
(	0.05	,	0.25	)
(	0.048	,	0.24	)
(	0.046	,	0.23	)
(	0.04	,	0.2	)
(	0.032	,	0.16	)
(	0.03	,	0.15	)
(	0.028	,	0.14	)
(	0.026	,	0.13	)
(	0.02	,	0.1	)
(	0	,	0	)
(	0.006	,	-0.03	)
(	0.008	,   -0.04	)
(	0.01	,	-0.05	)
(	0.012	,	-0.06	)
(	0.02	,	-0.1	)
(	0.026	,	-0.13	)
(	0.028	,	-0.14	)
(	0.03	,	-0.15	)
(	0.032	,	-0.16	)
(	0.04	,	-0.2	)
(	0.046	,	-0.23	)
(	0.048	,	-0.24	)
(	0.05	,	-0.25	)
(	0.052	,	-0.26	)
(	0.06	,	-0.3	)
(	0.08	,	-0.4	)
(	0.1	,	-0.5	)
(	0.12	,	-0.6	)
(	0.14	,	-0.7	)
(	0.16	,	-0.8	)
(	0.18	,	-0.9	)
(	0.2	,	-1	)
(	0.22	,	-1.1	)
(	0.24	,	-1.2	)
(	0.26	,	-1.3	)
(	0.28	,	-1.4	)
(	0.3	,	-1.5	)
(	0.32	,	-1.6	)
(	0.34	,	-1.7	)
(	0.36	,	-1.8	)
(	0.38	,	-1.9	)
(	0.4	,	-2	)
(	0.6	,	-3	)
(	0.8	,	-4	)
(	1	,	-5	)
(	1.2	,	-6	)
(	1.4	,	-7	)
(	1.6	,	-8	)
(	1.8	,	-9	)
(	2	,	-10	)
(	4	,	-20	)
(	6	,	-30	)
(	8	,	-40	)
(	10	,	-50	)
(	100	,	-500)

   
   
(	50	,	-500)
(	5	,	-50	)
(	4	,	-40	)
(	3	,	-30	)
(	2	,	-20	)
(	1	,	-10	)
(	0.9	,	-9	)
(	0.8	,	-8	)
(	0.7	,	-7	)
(	0.6	,	-6	)
(	0.5	,	-5	)
(	0.4	,	-4	)
(	0.3	,	-3	)
(	0.2	,	-2	)
(	0.19	,	-1.9	)
(	0.18	,	-1.8	)
(	0.17	,	-1.7	)
(	0.16	,	-1.6	)
(	0.15	,	-1.5	)
(	0.14	,	-1.4	)
(	0.13	,	-1.3	)
(	0.12	,	-1.2	)
(	0.11	,	-1.1	)
(	0.1	,	-1	)
(	0.09	,	-0.9	)
(	0.08	,	-0.8	)
(	0.07	,	-0.7	)
(	0.06	,	-0.6	)
(	0.05	,	-0.5	)
(	0.049	,	-0.49	)
(	0.048	,	-0.48	)
(	0.047	,	-0.47	)
(	0.046	,	-0.46	)
(	0.045	,	-0.45	)
(	0.044	,	-0.44	)
(	0.043	,	-0.43	)
(	0.042	,	-0.42	)
(	0.041	,	-0.41	)
(	0.04	,	-0.4	)
(	0.039	,	-0.39	)
(	0.038	,	-0.38	)
(	0.037	,	-0.37	)
(	0.036	,	-0.36	)
(	0.035	,	-0.35	)
(	0.034	,	-0.34	)
(	0.033	,	-0.33	)
(	0.032	,	-0.32	)
(	0.031	,	-0.31	)
(	0.03	,	-0.3	)
(	0.026	,	-0.26	)
(	0.025	,	-0.25	)
(	0.024	,	-0.24	)
(	0.023	,	-0.23	)
(	0.02	,	-0.2	)
(	0.016	,	-0.16	)
(	0.015	,	-0.15	)
(	0.014	,	-0.14	)
(	0.013	,	-0.13	)
(	0.01	,	-0.1	)
(	0.006	,	-0.06	)
(	0.005	,	-0.05	)
(	0.004	,   	-0.04	)
(	0.003	,	-0.03	)
(	0	,	0	)
(	0.01	,	0.1	)
(	0.013	,	0.13	)
(	0.014	,	0.14	)
(	0.015	,	0.15	)
(	0.016	,	0.16	)
(	0.02	,	0.2	)
(	0.023	,	0.23	)
(	0.024	,	0.24	)
(	0.025	,	0.25	)
(	0.026	,	0.26	)
(	0.03	,	0.3	)
(	0.04	,	0.4	)
(	0.05	,	0.5	)
(	0.06	,	0.6	)
(	0.07	,	0.7	)
(	0.08	,	0.8	)
(	0.09	,	0.9	)
(	0.1	,	1	)
(	0.11	,	1.1	)
(	0.12	,	1.2	)
(	0.13	,	1.3	)
(	0.14	,	1.4	)
(	0.15	,	1.5	)
(	0.16	,	1.6	)
(	0.17	,	1.7	)
(	0.18	,	1.8	)
(	0.19	,	1.9	)
(	0.2	,	2	)
(	0.3	,	3	)
(	0.4	,	4	)
(	0.5	,	5	)
(	0.6	,	6	)
(	0.7	,	7	)
(	0.8	,	8	)
(	0.9	,	9	)
(	1	,	10	)
(	2	,	20	)
(	3	,	30	)
(	4	,	40	)
(	5	,	50	)
(	50	,	500	)

 };   
    \path[fill=black] (0.04,1) circle (0.5pt) node[below]{\tiny{Q=10}};
        \path[fill=black] (0.21,0.98) circle (0.5pt) node[below]{\tiny{Q=5}};
    \end{smithchart}

   \begin{smithchart}
    [ticklabel style = {font=\tiny},
    show origin,
    width=14cm,
    grid style={red!60},
    ticklabel style={red!60},
                     ]
   \end{smithchart}
   \end{tikzpicture}
   \caption{Curva de Q en la carta de smith}
   \end{figure}
   
   
 %curva 2 ejemplo(falta en el doc)
\begin{figure}[ht!]
\centering
\begin{tikzpicture}[scale=0.8]
    \begin{polaraxis}[
                      width=15cm,
                      xshift=-0.5cm, 
                      yshift=-0.5cm,
                      xticklabel style={
                          sloped like x axis={
                              execute for upside down={\tikzset{anchor=north}},
                              reset nontranslations=false,
                          },
                          anchor=south,
                      },
                      xticklabel={\small\pgfmathprintnumber{\tick}\si{\degree}},
                      xtick align=center,
                      grid=none,
                      axis y line = none,]    
   \end{polaraxis}
   
   	\begin{smithchart}
   	[ticklabel style = {font=\tiny},
    width = 14cm,
   % show origin,
    smithchart mirrored,
    grid style={blue!60},
    ticklabel style={blue},  
     xticklabel shift=-10pt,
     ]
    %Curva Admitancia
\addplot[mark=none,<-,line width=2, smooth]
       coordinates{ 
(0.05,-1)(0.05,-0.9)(0.05,-0.8)(0.05,-0.75)(0.05,-0.7)(0.05,-0.65)(0.05,-0.6)(0.05,-0.5)(0.05,-0.4)
(0.05,-0.3)(0.05,-0.2)(0.05,-0.1)(0.05,-0)
};
\addplot[mark=none,->,line width=2, smooth]
       coordinates{ 
(1,4.7)(1,4.2)(1,3.8)(1,3.5)(1,3)(1,2.5)(1,2.0)(1,1.7)(1,1.5)(1,1.3)(1,1)(1,0.95)(1,0.9)(1,0.8)(1,0.7)(1,0.6)(1,0.5)(1,0.4)(1,0.3)(1,0.2)(1,0.1)(1,0)
};
      %Puntos de Acople
     \path[draw=black,fill=purple] (1,4.8) circle (2pt);
    \path[fill=black] (1.5,4.8) node[below]{\large{C}};
    \end{smithchart}

   \begin{smithchart}
    [ticklabel style = {font=\tiny},
    show origin,
    width=14cm,
    grid style={red!60},
    ticklabel style={red!60},
                     ]
        %Curva Impedancia
    \addplot[mark=none,line width=1.5,,color=orange, dotted]
       coordinates{
(	0	,	0	)(	0.0015	,	-0.03	)(	0.002	,  	-0.04	)(	0.0025	,	-0.05	)(	0.003	,	-0.06	)(	0.005	,	-0.1	)(	0.0065	,	-0.13	)(	0.007	,	-0.14	)(	0.0075	,	-0.15	)(	0.008	,	-0.16	)(	0.01	,	-0.2	)(	0.0115	,	-0.23	)(	0.012	,	-0.24	)(	0.0125	,	-0.25	)(	0.013	,	-0.26	)(	0.015	,	-0.3	)(	0.02	,	-0.4	)(	0.025	,	-0.5	)(	0.03	,	-0.6	)(	0.035	,	-0.7	)(	0.04	,	-0.8	)(	0.045	,	-0.9	)(	0.05	,	-1	)(	0.055	,	-1.1	)(	0.06	,	-1.2	)(	0.065	,	-1.3	)(	0.07	,	-1.4	)(	0.075	,	-1.5	)(	0.08	,	-1.6	)(	0.085	,	-1.7	)(	0.09	,	-1.8	)(	0.095	,	-1.9	)(	0.1	,	-2	)(	0.15	,	-3	)(	0.2	,	-4	)(	0.25	,	-5	)(	0.3	,	-6	)(	0.35	,	-7	)(	0.4	,	-8	)(	0.45	,	-9	)(	0.5	,	-10	)(	1	,	-20	)(	1.5	,	-30	)(	2	,	-40	)(	2.5	,	-50	)(	25	,	-500	)

 };   
    \addplot[mark=none,line width=1,,color=orange]
       coordinates{ 
(	0	,	0	)(	0.0015	,	0.03	)(	0.002	,  	0.04	)(	0.0025	,	0.05	)(	0.003	,	0.06	)(	0.005	,	0.1	)(	0.0065	,	0.13	)(	0.007	,	0.14	)(	0.0075	,	0.15	)(	0.008	,	0.16	)(	0.01	,	0.2	)(	0.0115	,	0.23	)(	0.012	,	0.24	)(	0.0125	,	0.25	)(	0.013	,	0.26	)(	0.015	,	0.3	)(	0.02	,	0.4	)(	0.025	,	0.5	)(	0.03	,	0.6	)(	0.035	,	0.7	)(	0.04	,	0.8	)(	0.045	,	0.9	)(	0.05	,	1	)(	0.055	,	1.1	)(	0.06	,	1.2	)(	0.065	,	1.3	)(	0.07	,	1.4	)(	0.075	,	1.5	)(	0.08	,	1.6	)(	0.085	,	1.7	)(	0.09	,	1.8	)(	0.095	,	1.9	)(	0.1	,	2	)(	0.15	,	3	)(	0.2	,	4	)(	0.25	,	5	)(	0.3	,	6	)(	0.35	,	7	)(	0.4	,	8	)(	0.45	,	9	)(	0.5	,	10	)(	1	,	20	)(	1.5	,	30	)(	2	,	40	)(	2.5	,	50	)(	25	,	500	) };
\addplot[mark=none,<-,line width=2, smooth]
       coordinates{ 
(0.042,0.195)(0.045,0.15)(0.045,0.1)(0.045,0.05)(0.045,0)(0.045,-0.1)(0.045,-0.2)(0.045,-0.3)(0.045,-0.4)(0.045,-0.5)(0.045,-0.6)(0.045,-0.7)(0.045,-0.75)(0.045,-0.8)(0.045,-0.85)(0.045,-0.9)(0.05,-0.99)
};

    \path[fill=black] (0.04,1) circle (0.5pt) node[below]{\tiny{Q=20}};
    
    
      %Puntos de Acople
     \path[draw=black,fill=purple] (20,0) circle (2pt);
    \path[fill=black] (28,0) node[below]{\large{A}};
    
    \path[draw=black,fill=purple] (1,0) circle (2pt);
    \path[fill=black] (1,0) circle (0.5pt) node[below]{\large{D}};
    
    \path[draw=black,fill=purple] (0.05,-1) circle (2pt);
    \path[fill=black] (0.05,-1) circle (0.5pt) node[above]{\large{B}};
                     
   \end{smithchart}
   \end{tikzpicture}
   \caption{Curva de acople ejemplo}
   \end{figure}
   
   
 %curva 3 ya esta en el documento  
\begin{figure}[ht!]
\centering
\begin{tikzpicture}
    \begin{polaraxis}[
                      width=15cm,
                      xshift=-0.5cm, 
                      yshift=-0.5cm,
                      xticklabel style={
                          sloped like x axis={
                              execute for upside down={\tikzset{anchor=north}},
                              reset nontranslations=false,
                          },
                          anchor=south,
                      },
                      xticklabel={\small\pgfmathprintnumber{\tick}\si{\degree}},
                      xtick align=center,
                      grid=none,
                      axis y line = none,]    
   \end{polaraxis}
   	\begin{smithchart}
   	[ticklabel style = {font=\tiny},
    width = 14cm,
   % show origin,
    smithchart mirrored,
    grid style={blue!60},
    ticklabel style={blue},  
     xticklabel shift=-10pt,
     ]
    %Curva Admitancia
    \addplot[mark=none,->,line width=1]
       coordinates{
           (0.5, 0)(0.5, 0.1)(0.5,0.2)(0.5,0.3)(0.5,0.4)(0.5,0.5)
           (0.5,0.6)(0.5,0.7)(0.5,0.8)(0.5,0.85)
           };
    \draw(0,2) node [below,color=black]{\tiny\bf{$0.9$}};

    \end{smithchart}

   \begin{smithchart}
    [ticklabel style = {font=\tiny},
    show origin,
    width=14cm,
    grid style={red!60},
    ticklabel style={red!60},
                     ]
    %Puntos de Acople
     \path[draw=black,fill=purple] (2,0) circle (1pt);
    \path[fill=black] (2,0) circle (0.5pt) node[below]{\tiny{A}};
    
    \path[draw=black,fill=purple] (0.5,0) circle (1pt);
    \path[fill=black] (0.5,0) circle (0.5pt) node[below]{\tiny{C}};
    
    \path[draw=black,fill=purple] (0.5,0.87) circle (1pt);
    \path[fill=black] (0.5,0.87) circle (0.5pt) node[above]{\tiny{B}};
    
    %Curva Impedancia                 
    \addplot[mark=none,->,line width=1]
      coordinates{
      
      (0.5,0.87) (0.5,0.8)
      (0.5,0.7) (0.5,0.6)
           (0.5,0.5) (0.5,0.4) (0.5,0.3) (0.5,0.28)          (0.5,0.25)(0.5,0.20)(0.5,0.18)(0.5,0.15)(0.5,0.13)(0.5,0.10)(0.5,0.08)(0.5,0.06)(0.5,0.03)(0.5,0)
          };
    \draw((0,0.9) node [below,color=black]{\tiny\bf{$0.85$}};
           
   \end{smithchart}
   
   \end{tikzpicture}
   \caption{Curvas de acople para el ejemplo 2.12}
   \end{figure}

\begin{ejem}{Bobina}{mittelwertsatz_n3}
	Es sei $n\in\mathbb{N}$, $D\subseteq\mathbb{R}^n$ eine offene Menge und$f\in C^{1}(D,\mathbb{R})$. Dann gibt es auf jeder Strecke$[x_0,x]\subset D$ einen Punkt $\xi\in[x_0,x]$, so dass gilt\begin{equation*}f(x)-f(x_0) = \operatorname{grad} f(\xi)^{\top}(x-x_0)\end{equation*}
\end{ejem}


\end{document}